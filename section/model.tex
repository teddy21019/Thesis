\subsection{Basis Structure}

The model is composed of two open economies and three sectors --- buyers/sellers
banks, and government. Each country issues its own currency, but all buyers have
the freedom to decide what means of payment (hereafter MoP) to use.  There is
an alternation of roles between buyers and sellers, with each transitioning to
the other role once a successful trade is conducted. This setting is to assure
that agents have the incentives to adjust its portfolio of MoP, conceptually
similar to the idea given in \citet*{TW95}. For each period, buyers decide how
much to consume and save, and base on the consumption budget, buyers meet with
sellers during a search and matching process. 

MoP is then decided under each trade. Both buyers and sellers observe the
popularity of each Mop, hence deciding the optimal portfolio of MoP to hold.
%%%%%%%%%%%%%%% 
\footnote{Under rational expectations, a representative agent looks forward and
chooses the optimal mean of payment that provides one the largest lifetime
utility. In the absence of perfect coordination and perfect foresight, however,
an agent might possibly hold a depreciating currency solely due to the fact that
it is the only means of payment widely used regionally.}
%%%%%%%%%%%%%%%
For every trade and portfolio reallocation that involves altering the banks' ledger, it is immediately recorded, and this in turn causes the leverage of the bank to alter. The leverage of the bank is globally visible to all agents, signalling the soundness of the financial environment. Sensitive agents are then urged to withdraw any premature assets from the bank (in this model I consider only the deposit) if they sense a signal of instability, and through a herding behavior tha bank is thus exposed to a risk of run. This herding behavior can be modeled through introducing an imitation rule \citep*{Santos2021}. For simplicity, the bank has an exogenous credit level. Doing so allows the result to be focused on the effect of cross-border CBDC, instead of other financial acceleration coming from the capital market \citep{BGG96}.  




