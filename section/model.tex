\subsection{Basis Flow and Structure}

The model is composed of two open economies and three sectors --- buyers/sellers
banks, and government. Each country issues its own currency, but all buyers have
the freedom to decide what means of payment (hereafter MoP) to use.  There is
an alternation of roles between buyers and sellers, with each transitioning to
the other role once a successful trade is conducted. This setting is to assure
that agents have the incentives to adjust its portfolio of MoP, conceptually
similar to the idea given in \citet*{TW95}. For each period, buyers decide how
much to consume and save, and base on the consumption budget, buyers meet with
sellers during a search and matching process.

MoP is then decided under each trade. Both buyers and sellers observe the
popularity of each Mop, hence deciding the optimal portfolio of MoP to hold.
%%%%%%%%%%%%%%%
\footnote{Under rational expectations, a representative agent looks forward and
chooses the optimal mean of payment that provides one the largest lifetime
utility. In the absence of perfect coordination and perfect foresight, however,
an agent might possibly hold a depreciating currency solely due to the fact that
it is the only means of payment widely used regionally.}
%%%%%%%%%%%%%%%
For every trade and portfolio reallocation that involves altering the banks'
ledger, the bank immediately records it. This in turn causes the leverage of the
bank to alter. The leverage of the bank is globally visible to all agents,
signalling the soundness of the financial environment. Sensitive agents are then
urged to withdraw any premature assets from the bank (in this model I consider
only the deposit) if they sense a signal of instability, and through a herding
behavior that bank is thus exposed to a risk of run. This herding behavior can be
modeled through introducing an imitation rule \citep*{Santos2021}. For
simplicity, the bank has an exogenous credit level. Doing so allows the result
to be focused on the effect of cross-border CBDC, instead of other financial
acceleration coming from the capital market \citep{BGG96}.

Central banks interest rates are temporarily set exogenously.
%% currently unclear whether the simple setting could cause inflation, and thus the Taylor rule is not compatible.



%%%%%%%%%%%%%%%%%%%%%%%%%%%%%%%%%%%%%%%%%%%%%%%%%%
%%
%%
%%
%%
%%
%Consumption Budget Decision
%%
%%
%%
%%
%%%%%%%%%%%%%%%%%%%%%%%%%%%%%%%%%%%%%%%%%%%%%%%%%%

\subsection{Consumption Decision}

Following \citet*{HandbookABM}, consumption decision for a buyer is separated into two steps: consumption budget and consumption bundle.

\subsubsection*{Consumption Budget}

We follow the simple keynesian setting and let the consumption be a proportion of its disposable income

\begin{definition}{General Rule of Consumption}

   For each buyer $b$ at time $t$, one consumes
   \begin{equation}
      C_{b,t} = \mathrm{MPC}\times \mathrm{DI}_{b,t} + a + \epsilon_t
   \end{equation}
   Where $\mathrm{MPC}$ is the marginal propensity to consume, $\mathrm{DI}$ is the disposable income, $a$ is the minimal amount of expenditure, and $\epsilon$ is a random shock on consumption.
\end{definition}

After all buyers complete deciding its consumption budget, they proceed to decide the consumption bundle.

\begin{definition}
{Consumption Bundle Step}
\begin{enumerate}
   \item Buyer $b$ select candidate sellers $\{s \in \mathcal{N}_b\}$ from its trading network $\mathcal{N}_b$.\footnote{See Appendix A for detail on the network}
   \item If seller $s$ is out of stock, proceed to the next candidate.
   \item If buyer $b$ itself is out of budget, one stops the consumption and the next buyer consumes.\footnote{In this agent-based model, buyers are activated randomly, but in order. In computer science this is called ``synchronous''}
   \item During each trade, the agents go through a payment decision process.
\end{enumerate}
\end{definition}

The payment decision process (hereafter PDP) depends on the frequency of past appearance, its opportunity cost of using it as a MoP (interest rate), its cost of holding the asset (storing cost or inflation rate), as depicted in the functionalism theory. Holding a MoP has different incentives from a payment aspect and an asset aspect.




