\subsection{Financial Stability of CBDC}

The literature on the financial disintermediation cause by CBDC is growling rapidly.
\citet*{FVSSU21} use a \citet*{DD83} framework to show that central bank has the
monopolistic power in the deposit market that can endanger the maturity
transformation for commercial banks.
\citet*{Keister19} uses a new monetary search model to prove that while CBDC increases the efficiency in exchange, it inevitably crowd out bank deposits, and cause the funding cost of a commercial bank to increase.

\begin{center}
(To be continued)
\end{center}

\subsection{CBDC in an Open Economy}
CBDC in an open economy is gaining large attention recently. \citet*{FMS22} proves that the international spillover effect might be amplified using a two-country DSGE model.

\begin{center}
(To be continued)
\end{center}

The closest model that addresses the issue of bank runs under cross-border CBDC
design is written by \citet{Popescu22}. Following \citet*{DD83}, the paper tried
to show that there are multiple equilibria of domestic bank is exposed in the
risk of a run. The model, however, only sees CBDC as a safe deposit, which is an
asset with interest rate, being a liability of the central bank. In reality,
agents might not want to deposit money in CBDC accounts solely due to it being
riskless, as it might not be the mean of payment widely used. This motivates the
methodology of using agent-based modelling, since it is capable of observing the
endogenous formation of CBDC as a means of payment, while simultaneously
activates an episode of fear through sentimental spreading.

\subsection{Agent-based Modeling}

\begin{center}
    Agent-based modeling intro
\end{center}

To motivate endogenous adoption of a certain means of payment, it is necessary
to embed a trading system. Several literatures proposes the designs of this
system. The Complex Adaptive Trivial System (CATS) mentioned in
\citet*{HandbookABM}, agents (households) choose firms from the lowest price in
the market during a consumption bundle decision stage; also in
\citet*{HandbookABM} the two framework EUBI and EBGE chooses the goods to buy
according to a logistic probability depending on the price firms offer. As for
the decision process of the selection of which means of payment to use during a
trade, these macroeconomic agent-based model does not implement this as they
focus more on quantitatively matching the stylized fact of a business cycle.
\citet*{Sargent90-MoneyAI} studies a \citet*{KW98} economy using a classifier
system evolving genetically to study the emergence of a medium of exchange.
\citet*{Manolis21} transforms an ABM model for language spreading into a study
for parallel circulation of currencies under a scale-free social
network.\footnote{In \citet*{Manolis21}, he generates a trading network using
preferential attachment, and according to network literatures, this network has
the property known as ``scale-free''\citep*{Price1976}. As you shall see in my model, I also
follow this idea to create a synthetic scale-free bipartite network as my
international trading net.}
His model, however, does not study the formation during trading process, but through
simply a spreading process.
A more structural model that studies the selection of a payment system during a repetitive trading process is \citet*{Trade_ABM_MOP2006}. They study how agents adopt to new payments systems using ABM, and the probability of selecting a means of payment depends on the utility function of selecting it.

Although the literature of ABM is abundant regarding the trading process, there is a crucial component that is missing, which unfortunately is the key to my research --- the expectation for events. For macroeconomic agent-based models, the expectation can be addressed by artificial neural networks \cite[see][]{Salle15-ABM_EXP} or genetic algorithms \cite[see][]{Arifovic18-GA_EXP}.