\subsection{Financial Stability of CBDC}

The literature on the financial disintermediation cause by CBDC is growling rapidly.
\citet*{FVSSU21} use a \citet*{DD83} framework to show that central bank has the
monopolistic power in the deposit market that can endanger the maturity
transformation for commercial banks.
\citet*{Keister19} uses a new monetary search model to prove that while CBDC increases the efficiency in exchange, it inevitably crowd out bank deposits, and cause the funding cost of a commercial bank to increase.

\begin{center}
(To be continued)
\end{center}

\subsection{CBDC in an Open Economy}
CBDC in an open economy is gaining large attention recently. \citet*{FMS22} proves that the international spillover effect might be amplified using a two-country DSGE model.

\begin{center}
(To be continued)
\end{center}

The closest model that addresses the issue of bank runs under cross-border CBDC
design is written by \citet{Popescu22}. Following \citet*{DD83}, the paper tried
to show that there are multiple equilibria of domestic bank is exposed in the
risk of a run. The model, however, only sees CBDC as a safe deposit, which is an
asset with interest rate, being a liability of the central bank. In reality,
agents might not want to deposit money in CBDC accounts solely due to it being
riskless, as it might not be the mean of payment widely used. This motivates the
methodology of using agent-based modelling, since it is capable of observing
both an endogenous formation of CBDC as a means of payment, while simultaneously
activates an episode of fear through sentimental spreading.


\subsection{Agent-based Modeling}
