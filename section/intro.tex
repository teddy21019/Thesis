During the outbreaks of Covid-19, physical cash and coins are considered to be
one of the possibilities of invisible route of transmission, and more and more
individuals choose to pay with digital payments. 
(introduction of CBDC)
One major consideration is its risk of financial disintermediation[cite].  As it
allows people to hold an ``account'' directly in the central bank, people used
to digital payment have incentives to transfer their cash and deposit into
holding CBDC if it offers the same interest rate[cite], since CBDC is by
definition backed by central bank, and typically has no risk of run. The
withdrawal might cause commercial banks to shrink their balance sheet, hence
increasing their leverage on credits.  The signal of high leverage and low money
holding further causes withdrawal from non-digital payment users, and the
economy enters a bank-run episode. 

As mentioned by \citet*{DD83}, measures like deposit insurance can be used to
alleviate the fear of a bank run, hence preventing bank runs to happen. However,
this will fail to be the solution if the CBDC that is held widely in the current
country is issued by a foreign country. When the foreign currency is also a
potential candidate of a global means of payment, this currency substitution
effect will be especially difficult to prevent. For potential agents about to withdraw, deposit insurance is no longer a
convincing solution, since central banks may also run out of foreign reserves. 

As dreadful as it looks, currency substitution might not necessarily happen.
Decisions of means of payment is a global coordination game through a
self-fulfilling process[cite]. 