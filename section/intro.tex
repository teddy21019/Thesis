During the outbreaks of Covid-19, physical cash and coins are considered to be
one of the possibilities of invisible route of transmission, and more and more
individuals choose to pay with digital payments.
(introduction of CBDC)
One major consideration is its risk of financial disintermediation[cite].  As it
allows people to hold an ``account'' directly in the central bank, people used
to digital payment have incentives to transfer their cash and deposit into
holding CBDC if it offers the same interest rate[cite], since CBDC is by
definition backed by central bank, and typically has no risk of run. The
withdrawal might cause commercial banks to shrink their balance sheet, hence
increasing their leverage on credits.  The signal of high leverage and low money
holding further causes withdrawal from non-digital payment users, and the
economy enters a bank-run episode.

As mentioned by \citet*{DD83}, measures like deposit insurance can be used to
alleviate the fear of a bank run, hence preventing bank runs to happen. However,
this will fail to be the solution if the CBDC that is held widely in the current
country is issued by a foreign country. When the foreign currency is also a
potential candidate of a global means of payment, this currency substitution
effect will be especially difficult to prevent.
For potential agents about to withdraw, deposit insurance is no longer a
convincing solution of stability, since central banks may also run out of foreign reserves.

As dreadful as it looks, currency substitution might not necessarily happen.
Decisions of means of payment is a global coordination game through a
self-fulfilling process[cite], and without the foreign CBDC being widely
utilized, the episode is unlikely to happen.
As most literatures in CBDC considers the policy and welfare implications in an
economy purely embedded with CBDC, the era where physical cash and CBDC coexists
might not be happening so quickly, and countries with different levels of
development might also adapt to this new system in different paces, therefore the
scenario of currency substitution is worth considering.
In this paper, I deploy an Agent-Based Model (ABM) to fill the gap of the past
literature by considering the effect of the adoption of a mean of payment. The
decision of means of payment for each individual can be dynamically observed,
and it is closely determined by the network and interaction protocol \citep*{KW98}.

The model consists of three sectors: buyer sector, seller sector, and commercial
bank sector. The buyers and sellers toggle to each other periodically, so that
the interaction strategically complements the popularity of some means of
payments, reinforcing (and as well suppressing) the adoption of CBDC. The
usage of a certain means of payment thus determines their portfolio decision,
and hence affecting the balance sheet of a commercial bank. The balance sheet as
well as its leverage ratio is publicly observed by all agents, and agents with
difference tolerance rate react and withdraw accordingly. The fear is spread
along the social network, potentially causing a bank run.

The key different between CBDC and other existing means of payment is that CBDC
can be designed to have high interoperability, where substitution from domestic
bank deposit to foreign CBDC is more effortless and cheaper compared to foreign
bank deposits, as traditional cross border payment usually involves a multitude
of intermediates. See table~\ref{table: compare-mop}.

\begin{table}[t]
    \centering
    \begin{tabular}{c c c c c c}
        &H-Cash&F-Cash&H-Deposit&F-Deposit& F-CBDC \\
        \hline
        Return & $1$ & $1$ & $i_{HD}$ & $i_{FD}$ & $i_{CBDC}$ \\
        Risk & Low & Low & High & High & Low \\
        Accessibility & High & Medium & High & Low & High \\
        \hline
    \end{tabular}
    \caption{Comparison for different asset/means of payments. The evaluation is
    done from a home-country agent's perspective. Prefix H represents ``Home'';
    prefix F represents ``Foreign''.}
    \label{table: compare-mop}
\end{table}

The rest of this paper is organized as follows. In section \# I review some
literatures regarding the financial implications of CBDC and a brief
introduction to ABM. Section \# describes the model in detail. Section \#
demonstrates the simulation result with some sensitivity check. Finally, I
conclude the policy implications and future extensions regarding this issue.